\documentclass[12pt]{article}

\usepackage[T1]{fontenc}
\usepackage[utf8]{inputenc}
\usepackage[english, russian]{babel}
\usepackage[paperheight=29.7cm, paperwidth=21.0cm, left=2cm, right=2cm, top=1.5cm, bottom=1.5cm]{geometry}
\usepackage[hidelinks]{hyperref}
\usepackage{multicol}
\usepackage[neverdecrease]{paralist}
\usepackage{xcolor}

\setlength\parindent{0pt}
\setlength{\columnsep}{1.5cm}
\setdefaultleftmargin{0mm}{}{}{}{}{}
\pagestyle{empty}

\begin{document}

\begin{center}
{\Large \textbf{Лунные и Солнечные Затмения. Задачи}}
\end{center}
\begin{center}
{\Large 7 февраля 2022 @iaa2005}
\end{center}

\begin{enumerate}
	\item Луна в фазе последней четверти покрывает (то есть заслоняет от наблюдателя своим диском) звезду Альдебаран в созвездии Тельца.
	\begin{enumerate}
 	\item В какое время суток можно наблюдать это покрытие?
 	\item У какого края диска Луны произойдёт покрытие: у освещённого или затенённого?
 	\item Предположим, в следующем месяце случится ещё одно покрытие той же звезды. Как изменится фаза Луны: увеличится или уменьшится?
 	\item Можно ли будет хотя бы одно из этих покрытий наблюдать на южном полюсе Земли?
 	\end{enumerate}
 	
 	\item В первый день нового года, $1$ января, Луна оказалась в фазе полнолуния, одновременно на ее диске наблюдались максимальные либрации по широте к югу и по долготе к западу, то есть наилучшим возможным образом были видны участки южного и западного полушарий Луны. В какую дату начавшегося года можно ожидать полное солнечное затмение? Драконический месяц (период между двумя прохождениями Луны одного узла своей орбиты) составляет $27.2122$ суток.
 	\item Два покрытия Марса Луной произошли с интервалом $26.5$ суток. Какой (примерно) была фаза Луны во время покрытий? (Н. Шатовская)
 	\item В феврале $2015$ года на Земле началась серия ежемесячных покрытий звезды Альдебаран ($\alpha$ Тельца) Луной. Каждое покрытие видно из разных областей Земли. Эклиптическая широта Альдебарана составляет $-5.47^{\circ}$. Определите, до какого времени будет продолжаться эта серия. Орбиту Луны считать круговой. (Всеросс-$2016$)
 	\item Полное затмение началось в $23.09$ в точке на Земле с координатами $(\lambda=0^{\circ};\varphi=0^{\circ})$. Определите, сколько еще саросов будет видна эта серия затмений.
 	\item $1$ июля $2011$ года в акватории между южной Африкой и Антарктидой произойдет частное солнечное затмение с небольшой фазой, которое откроет новую последовательность солнечных затмений (в предыдущий сарос, $20$ июня $1993$ года, затмения не произошло). У какого из узлов лунной орбиты произойдет это затмение? Оцените, до какого века будет продолжаться данная последовательность. Эксцентриситетом орбиты Луны пренебречь.
 	\item Далекая звезда находится на небе в точке летнего солнцестояния. Во время прохождения мимо звезды восходящего узла лунной орбиты на Земле в каждый оборот Луны будет наблюдаться покрытие данной звезды Луной. Сколько покрытий будет содержать серия? На какой широте и в какой части неба будет видно первое и последнее покрытие звезды Луной в серии? Орбиту Луны считать круговой.
 	\item Малая планета обращается вокруг Солнца по круговой орбите. Хотя плоскость этой орбиты совпадает с плоскостью эклиптики, на Земле никогда не наблюдаются покрытия данной планеты Луной. Определить радиус орбиты малой планеты.
 	\item Звезда Эль-Натх в каталоге Иоганна Байера  получила два обозначения: $\beta$ Тельца и $\gamma$ Возничего. На Земле во время полнолуния наблюдается покрытие Луной этой звезды. Определите, можно ли в течение месяца после этого покрытия наблюдать с Земли лунное или солнечное затмение? В каком созвездии окажется Луна во время ближайшего после этого покрытия новолуния?
 	\item Определите диапазон эклиптических широт звезд, для которых возможно покрытие Венерой? (Наклон орбиты Венеры к плоскости эклиптики $3.4^{\circ}$)
\end{enumerate}

\end{document}

